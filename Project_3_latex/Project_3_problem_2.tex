\documentclass[a4paper,11pt]{article}
\usepackage{graphicx}
\usepackage{float}
\usepackage{subfig}
\usepackage{geometry}
\usepackage{amsmath,amssymb}
\usepackage{amsthm}
\usepackage{bbold}
\usepackage{mathtools}
\usepackage{braket}
\usepackage{booktabs}
\usepackage[table,xcdraw]{xcolor}
\usepackage[utf8]{inputenc}
\usepackage{cite}
\usepackage[english]{babel}
\usepackage{lipsum}
\usepackage{setspace}
%\usepackage{minted}
\usepackage{xcolor}
\newcommand{\R}{\mathbb{R}}
\usepackage{hyperref}
\hypersetup{colorlinks=true,linkcolor=blue}
\geometry{a4paper, top=2.5cm, bottom=2.5cm, left=3cm, right=2.5cm}

\begin{document}
	\author{Catalano Giuseppe, Cerrato Nunzia}
	\title{Numerical Linear Algebra Homework Project 3:\\Eigenvalues and Eigenvectors}
	\date{}
	\maketitle
	
\section*{Problem 2}	
Here we consider approximations to the eigenvalues and eigenfunctions of the one-dimensional Laplace operator 
$L[u] := - \frac{d^2 u }{dx^2}$ on the unit interval $[0,1]$ with boundary conditions $u(0) = u(1) = 0$. A scalar $\lambda$ is said to be an eigenvalue of $L$ (with homogeneous Dirichlet boundary conditions) if there exists a twice-differentiable function $u : [0, 1] \rightarrow \R$, not identically zero in $[0, 1]$, such that
\begin{equation}\label{eq: continuous differential eigenproblem}
	-u''(x) = \lambda u(x) \text{ on } [0,1] \text{ with } u(0) = u(1) = 0.
\end{equation}
In this case $u$ is said to be an \textit{eigenfunction} of $L$ corresponding to the eigenvalue $\lambda$. Obviously, eigenfunctions are defined up to a nonzero scalar multiple.
The eigenvalues and eigenfunctions of $L$ are easily found to be $\lambda_j = j^2\pi^2$ and $u_j(x) = \alpha \sin(j\pi x)$ for any nonzero constant $\alpha$, which we can take to be $1$. Here $j$ is a positive integer; hence, the operator $L$ has an infinite set of (mutually orthogonal) eigenfunctions $\{u_j\}_{j=1}^\infty $ corresponding to the discrete spectrum of eigenvalues ${\lambda_j}_{j=1}^\infty $. Note that $0 < \lambda_1 < \lambda_2 < \dots < \lambda_j \rightarrow \infty$ as $j \rightarrow \infty$. Also, each eigenvalue is simple in the sense that (up to a scalar multiple) there is a unique eigenfunction corresponding to it. Approximations to the eigenvalues and eigenfunctions can be obtained by discretizing the interval $[0, 1]$ bymeansof $N+2$ evenly spaced points: $x_i =ih \text{ where } i=0,1,...,N+1 \text{ and } h=1/(N+1)$. The second derivative operator can then be approximated by centered finite differences:
	
\begin{equation}
	-\frac{d^2u}{dx^2}(x_i) \approx \frac{-u(x_{i-1} + 2 u(x_i)  -2 u(x_{i+1}) )}{h^2}
\end{equation}
and therefore the continuous (differential) eigenproblem \eqref{eq: continuous differential eigenproblem} can be approximated by the discrete (algebraic) eigenvalue problem 

\begin{equation}\label{key}
	h^{-2} T_N \textbf{u} = \lambda \textbf{u}
\end{equation}
where we have set

\begin{equation}\label{key}
	T_N = \begin{bmatrix}
		2 & -1 &  & 0 \\
		-1 & \ddots  & \ddots  &  \\
		& \ddots & \ddots & -1 \\
		0 &  & -1 & 2
	\end{bmatrix}, \text{ and } \textbf{u} = \begin{bmatrix}
	u_1\\
	u_2\\
	\vdots\\
	u_N
\end{bmatrix} 
\end{equation}
with $u_i := u(x_i)$. It can be shown that the $N \times N$ matrix $T_N$ has eigenvalues $\mu_j = 2(1- \cos \frac{\pi j }{N+1})$ for $j = 1, \dots, N$, corresponding to the eigenvectors $\textbf{u}_j$, where $\textbf{u}_j(k) = \sqrt{\frac{2}{N+1}} \sin( \frac{j k \pi}{N+1} $) is the $kth$ entry in $\textbf{u}_j$. Notice that the eigenvectors $\textbf{u}_j$ are normalized with respect to the 2-norm: $\textbf{u}^T_j \textbf{u}_j = 1$. Also notice that the eigenvalues of $T_N$ lie in the interval $(0, 4)$. Hence, the eigenvalues of $h^{-2} T_N$ lie in the interval $(0, 4(N + 1)2 )$.\\


\noindent \textbf{(1)} Since we are considering $j\ll N$ and $N\gg1$ we can identify the Taylor expansion of $\cos x$ with $x = \frac{\pi j }{N+1}$:
\begin{equation}
	\cos \frac{\pi j }{N+1} = \cos \pi j h = 1 - \frac{1}{2} \pi^2 j^2 h^2 + O(h^4),
\end{equation}
that leads us to approximate the smallest eigenvalues of $2h^{-2} T_N$ as follows:
\begin{equation}\label{key}
	2h^{-2} (1- \cos \frac{\pi j }{N+1} ) = 2h^{-2} (1- 1 + \frac{1}{2} \pi^2 j^2 h^2 + O(h^4)) = \pi^2 j^2  + O(h^2) \simeq \pi^2 j^2,
\end{equation}
where we used that $h=1/N+1$.\\
\noindent For the largest eigenvalue of $T_N$, we have that $j = N$, therefore we can not truncate anymore the taylor expansion of the cosine if we want a good approximation. We can compute the $N-th$ eigenvalue of $T_N$ in the limit of $N\gg 1$:
\begin{equation}\label{key}
	\mu_N = 2(1-\cos\pi  \frac{N}{N+1})  =2(1-\cos (\pi -\pi h) )) = 2(1+cos\pi h) = 4 - \pi^2 h^2 + O(h^4)
\end{equation}
Therefore, we have
\begin{equation}\label{key}
	h^{-2} \mu_N = 4(N+1)^2 - \pi^2 + O(h^2)
\end{equation}
that is not a good approximation of $\lambda_N = \pi^2 N^2$ 

\noindent \textbf{(2)} We want to compare the eigenvectors $\textbf{u}_j$ of $T_N$ with the eigenfunctions of $L$, up to the normalization constant, that we will set to $1$ for both. If we recall that $x_k = k h\ \forall k = 1,\dots, N$ we can observe that the $k-th$ component of the eigenvector $\textbf{u}_j$ is equal to the $j-th$ eigenfunction $u_j(x)$ computed in corrispondence of the value $x=x_k$:
\begin{equation}
	u_j(x_k) = \sin(j \pi x_k) = \sin ( j \pi k h ) = \sin\left( \frac{j \pi k}{N+1}\right)  = \textbf{u}_j(k).
\end{equation}

\noindent \textbf{(3)}  Now we compute the spectral condition number of $T_N$ in the limit of $N\gg 1$. We recall that the eigenvalues of $T_N$ are
\begin{equation}\label{key}
	\mu_j = 2\left( 1-\cos \frac{\pi j}{N+1}\right) = 2\left( 1-\cos \pi j h\right)
\end{equation}
\begin{equation}
\begin{split}
	h^{-2}\mu_1 &= h^{-2} 2 \left(  1-1 +\frac{1}{2} \pi^2 h^2 - \frac{1}{4!} \pi^4 h^4 + O(h^6)\right) \\
	& = \pi^2 -\frac{1}{12} \pi^4 h^2 + O(h^4)
\end{split}
\end{equation}

\begin{equation}
\begin{split}
	h^-2\mu_N &= h^{-2}2(1+cos(\pi h))\\
	&=2h^{-2} \left( 1 + 1 -\frac{1}{2} \pi^2 h^2 + \frac{1}{4!} \pi^4 h^4 + O(h^6)\right) \\
	&= 4 h^{-2} - \pi^2 +\frac{1}{12} \pi^4 h^2 + O(h^4)
\end{split}
\end{equation}

\begin{equation}
\begin{split}
	k_2(T_N) &= \frac{h^{-2}\mu_N}{h^{-2}\mu_1} = \frac{4 h^{-2} - \pi^2 +\frac{1}{12} \pi^4 h^2 + O(h^4)}{\pi^2(1 -\frac{1}{12} \pi^2 h^2 + O(h^4))}\\
	&= \frac{4 h^{-2}}{\pi^2} -1 + \frac{4 h^{-2}}{\pi^2} \frac{\pi^2 h^2}{12} +O(h^2)\\
	&=\frac{4}{\pi^2}(N+1)^2 - \frac{2}{3} + O(N^{-2})
\end{split}
\end{equation}
	
	
\end{document}